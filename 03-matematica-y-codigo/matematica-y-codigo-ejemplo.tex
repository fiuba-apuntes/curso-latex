\documentclass{article}

\usepackage[utf8]{inputenc} % Indica cuál es la codificación de este archivo
\usepackage[spanish]{babel} % Indica el idioma en que está escrito el documento
\usepackage[svgnames]{xcolor} % Permite usar colores en el documento
\usepackage{amssymb} % Fuentes y símbolos adicionales (por AMS)
\usepackage{amsmath} % Mejoras a entornos matemáticos y extras (por AMS)
\usepackage{mathtools} % Correcciones a amsmath y funcionalidades extras
\usepackage{listings} % Permite ingresar código fuente de software

% Redefino algunos nombres para reemplazar los de Babel Spanish
\addto\captionsspanish{
  \renewcommand\lstlistingname{Código}
  \renewcommand\lstlistlistingname{Lista de código}
}

\lstset{ % Defino el formato de bloques de código fuente
  inputencoding=utf8, % Indico la codificación de los archivos de entrada
  extendedchars=true, % Extiendo los caracteres
  % Escapeo caracteres especiales
  literate={á}{{\'a}}1 {é}{{\'e}}1 {í}{{\'i}}1 {ó}{{\'o}}1 {ú}{{\'u}}1
           {Á}{{\'A}}1 {É}{{\'E}}1 {Í}{{\'I}}1 {Ó}{{\'O}}1 {Ú}{{\'U}}1
           {ä}{{\"a}}1 {ë}{{\"e}}1 {ï}{{\"i}}1 {ö}{{\"o}}1 {ü}{{\"u}}1
           {Ä}{{\"A}}1 {Ë}{{\"E}}1 {Ï}{{\"I}}1 {Ö}{{\"O}}1 {Ü}{{\"U}}1
           {ñ}{{\~n}}1 {Ñ}{{\~N}}1,
  inputpath={codigo}, % Defino la ruta en la que se encuentran los archivos
  showstringspaces=false, % Indica si muestra los espacios dentro de strings
  numbers=left, % Posición en que se muestran los números de línea
  numberstyle=\tiny\color{gray}, % Estilo de los números de línea
  breaklines=true, % Se parten las líneas que exceden del ancho del documento
  frame=single, % Formato del marco del entorno
  backgroundcolor=\color{gray!5}, % Color de fondo
  basicstyle=\ttfamily\footnotesize, % Estilo de base (familia, tamaño, color)
}

\lstdefinestyle{cpp}{
  tabsize=4,
  language=C++,
  keywordstyle=\color{DarkGreen}, % Estilo de las palabras reservadas
  stringstyle=\color{DarkBlue}, % Estilo de los strings
  commentstyle=\color{DarkGray}, % Estilo de los comentarios
  otherkeywords={std,cout} % Agrego palabras reservadas que no se resaltan
}

\lstdefinestyle{ruby}{
  language=Ruby,
  keywordstyle=\color{DarkMagenta}, % Estilo de las palabras reservadas
  stringstyle=\color{DarkBlue}, % Estilo de los strings
  commentstyle=\color{DarkGray} % Estilo de los comentarios
}

\begin{document}

\section{Fórmulas matemáticas}

\subsection{Entornos matemáticos}

\subsubsection{Entorno en línea}

La funcion $f(x)$ es $f(x)=ax+b$.

\subsubsection{Entorno ecuación (sin numerar)}
La ecuación

\[
  f(x)=ax+b
\]

es una función $f(x)$.

\[
  10 \text{ manzanas} + 10 \text{ manzanas} = 20 \text{ manzanas}
\]

\subsubsection{Entorno ecuación}

\begin{equation} \label{eq:ecuacion}
  f(x) = ax + b
\end{equation}

La ecuación \eqref{eq:ecuacion} es una función $f(x)$.

\subsection{Nombre de funciones, fuentes y símbolos}

\subsubsection{Algunos nombres de funciones}

\[
  \cos (2\theta) = \cos^2 \theta - \sin^2 \theta
\]


\subsubsection{Fuentes adicionales provistas por AMS}

\textbf{Letras ``Blackboard bold'' (sólo en mayúsculas)}

$\mathbb{ABCDEFGHIJKLMNOPQRSTUVWXYZ}$ \\

\textbf{Letras ``Euler Fraktur''}

$\mathfrak{abcdefghijklmnopqrstuvwxyz}$

$\mathfrak{ABCDEFGHIJKLMNOPQRSTUVWXYZ}$ \\

\textbf{Letras ``Euler Script'' (sólo en mayúsculas)}

$\mathcal{ABCDEFGHIJKLMNOPQRSTUVWXYZ}$

\subsubsection{Símbolos adicionales provistos por AMS}

$\boxdot, \boxminus, \boxplus, \boxtimes, \Cap, \Cup$

\subsection{Potencias, índices, raíces y fracciones}

\subsubsection{Potencias e índices}

\begin{equation*}
  k_{n+1} = n^2 + k_n^2 - k_{n-1}
\end{equation*}

\subsubsection{Raíces}

\begin{equation*}
  \sqrt{4} = 2
\end{equation*}

\begin{equation*}
  \sqrt[3]{8} = 2
\end{equation*}

\subsubsection{Fracciones}

\begin{equation*}
  \frac{\frac{1}{x}+\frac{1}{y}}{y-z}
\end{equation*}

\subsection{Matrices}

\begin{equation*}
  \begin{matrix}
    a & b & c \\
    d & e & f \\
    g & h & i
  \end{matrix}
\end{equation*}

Matriz encerrada entre paréntesis

\begin{equation*}
  \begin{bmatrix*}[r]
    -1 & 3  \\
    2  & -4
  \end{bmatrix*}
\end{equation*}

\begin{equation*}
  A_{m,n} = 
    \begin{pmatrix}
      a_{1,1} & a_{1,2} & \cdots & a_{1,n} \\
      a_{2,1} & a_{2,2} & \cdots & a_{2,n} \\
      \vdots  & \vdots  & \ddots & \vdots  \\
      a_{m,1} & a_{m,2} & \cdots & a_{m,n}
    \end{pmatrix}
\end{equation*}

\section{Colores}

\subsection{Colores disponibles por defecto de \texttt{xcolor}}

{\color{red} Rojo}

{\color{green} Verde}

{\color{blue} Azul}

{\color{cyan} Cyan}

{\color{magenta} Magenta}

{\color{yellow} Amarillo}

{\color{black} Negro}

{\color{darkgray} Gris oscuro}

{\color{gray} Gris}

{\color{lightgray} Gris claro}

{\color{white} Blanco} (Blanco)

{\color{brown} Marron}

{\color{lime} Lima}

{\color{olive} Oliva}

{\color{orange} Naranja}

{\color{pink} Rosa}

{\color{purple} Púrpura}

{\color{teal} Verde azulado}

{\color{violet} Violeta}

\subsection{Algunos colores adicionales incluidos por la opción \texttt{svgnames}}

{\color{Aquamarine} Agua marina}

{\color{Chocolate} Chocolate}

{\color{Crimson} Crimson}

{\color{GreenYellow} Verde amarillo}

{\color{DarkBlue} Azul oscuro}

{\color{LightBlue} Azul claro}

{\color{DarkCyan} Cyan oscuro}

{\color{LightCyan} Cyan claro}

{\color{DarkGreen} Verde oscuro}

{\color{LightGreen} Verde claro}

{\color{DarkMagenta} Magenta oscuro}

{\color{DarkOrange} Naranja oscuro}

{\color{DarkRed} Rojo oscuro}

{\color{Salmon} Salmón}

{\color{SkyBlue} Azul cielo}

{\color{Tomato} Tomate}

{\color{YellowGreen} Amarillo verde}

\subsection{Usando los colores}

\subsubsection{Formateo de texto}

{\color{red} Grupo de texto con color rojo}

\textcolor{blue}{Comando con color azul para texto}

\colorbox{green}{Texto con fondo verde}

\colorbox{black}{\color{white} Texto blanco con fondo negro}

\fcolorbox{red}{white}{Texto con borde de color rojo}

\fcolorbox{blue}{green}{Texto con borde de color azul y fondo verde}

\section{Código de software}

\subsection{Entorno \texttt{verbatim}}

\begin{verbatim}
/* Hello World program */

#include<stdio.h>

main() {
    printf("Hello World");
}
\end{verbatim}

\subsection{Paquete \texttt{listing}}

\subsubsection{Entorno \texttt{lstlisting}}

\begin{lstlisting}[caption={Código C},label={lst:codigoc}]
/* Hello World program */

#include<stdio.h>

main() {
    printf("Hello World");
}
\end{lstlisting}

\subsubsection{Importar código desde un archivo externo}

\textbf{Código de lenguaje C}

\lstinputlisting[
  language=C,
  caption={Código C importado},
  label={lst:codigocimportado}
]{main.c}

~\\

\textbf{Código de lenguaje C++}

\lstinputlisting[
  style=cpp,
  caption={Código C++ importado},
  label={lst:codigocpp}
]{main.cpp}

~\\

\textbf{Código de lenguaje Ruby}

\lstinputlisting[
  style=ruby,
  caption={Código Ruby},
  label={lst:codigoruby}
]{script.rb}

\subsubsection{Indice de códigos}
\lstlistoflistings

\end{document}
