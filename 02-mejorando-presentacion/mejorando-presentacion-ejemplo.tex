\documentclass{article}

\usepackage[utf8]{inputenc} % Indica cuál es la codificación de este archivo
\usepackage[spanish]{babel} % Indica el idioma en que está escrito el documento
\usepackage{graphicx}

\begin{document}

% Ingreso contenido de la caratula estándar de Latex
\title{Un documento de ejemplo} % Título del documento
\date{\today} % Fecha del documento. El comando \today inserta la fecha actual
\author{Ing. Huergo\thanks{FIUBA}} % Autor del documento

\maketitle % Genera la caratula

\newpage % Inicia una nueva página

\tableofcontents % Genera la tabla de contenido del documento

\newpage % Inicia una nueva página

\section*{Introducción}
\addcontentsline{toc}{section}{Introducción}

Con este ejemplo se aprende sobre LaTeX:

\begin{itemize}
  \item Tabla de contenidos (índices)
  \item Carátulas
  \item Gráficos y entorno de figuras
\end{itemize}

\section{Lorem ipsum dolor sit amet, consectetur adipiscing elit. Suspendisse ac.}

\subsection[Fusce in]{Lorem ipsum dolor sit amet, consectetur adipiscing elit. Fusce in.}

\section{Gráficos}

\subsection{Incluir un gráfico}

\begin{figure}[h]
  \centering
  \caption{Logo de la UBA}
  \includegraphics[width=0.6\textwidth]{logo_uba}
  \label{fig:logo_uba}
\end{figure}

Como se ve en la Figura \ref{fig:logo_uba}, el logo de la UBA posee el lema:

\begin{center}
  \textit{Argentum virtus robur et studium}

  <<La virtud argentina es la fuerza y el estudio>>
\end{center}

\listoffigures

\end{document}
